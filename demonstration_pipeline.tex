\documentclass[11pt, a4paper]{article}

% --- Packages ---
\usepackage[utf8]{inputenc}
\usepackage[T1]{fontenc}
\usepackage{geometry}
\usepackage{xcolor}
\usepackage{listings}
\usepackage{titlesec}
\usepackage{enumitem}
\usepackage{hyperref}
\usepackage{parskip} % Adds space between paragraphs

\geometry{margin=1in}

% --- Colors ---
\definecolor{primary}{RGB}{0, 51, 102}
\definecolor{codebg}{RGB}{245, 245, 245}
\definecolor{warning}{RGB}{204, 0, 0}

% --- Styling ---
\titleformat{\section}{\Large\bfseries\color{primary}}{}{0em}{}[\titlerule]
\titleformat{\subsection}{\large\bfseries\color{primary}}{}{0em}{}

\lstset{
    basicstyle=\ttfamily\small,
    backgroundcolor=\color{codebg},
    frame=single,
    breaklines=true,
    rulecolor=\color{gray!30},
    showstringspaces=false,
    keywordstyle=\color{blue},
    commentstyle=\color{green!50!black},
}

% --- Meta ---
\title{\textbf{Zero-Day Cyber Sentinel} \\ \Large Demonstration Pipeline}
\author{Documentation}
\date{\today}

\begin{document}

\maketitle

\textbf{Total Duration:} $\sim$ 2-3 Minutes \\
\textbf{Goal:} Demonstrate the real-time nature of the pipeline and the AI analysis value.

\section{Prerequisites}

Before starting the demo, ensure a clean state:

\begin{lstlisting}[language=bash]
# 1. Clear old data for a clean demo
rm stream.jsonl alerts.jsonl knowledge_base.jsonl 2>/dev/null

# 2. Start the system (Docker method)
docker build -t sentinel .
docker run -p 8501:8501 --env-file .env -v "${PWD}:/app" sentinel

# 3. Wait ~30 seconds for initial NIST data to populate
\end{lstlisting}

\section{Demo Flow}

\subsection{Step 1: Show the Architecture (30 seconds)}
Open the 3 terminals/logs running in Docker or your local environment:
\begin{enumerate}
    \item \textbf{Stream Generator}: Fetching NIST CVE data every 5 seconds.
    \item \textbf{Pathway Engine}: Processing and matching threats.
    \item \textbf{Streamlit Dashboard}: Real-time visualization.
\end{enumerate}

\subsection{Step 2: Show the Dashboard (30 seconds)}
Navigate to \url{http://localhost:8501}:
\begin{itemize}
    \item Point out the \textbf{Live NIST Threat Intelligence} table (raw stream).
    \item Point out the \textbf{Matched Alerts} section (currently empty = system is secure).
    \item Show the \textbf{Threat Detection Timeline} chart.
\end{itemize}

\subsection{Step 3: Inject a Custom Threat (1 minute)}
This is the ``magic moment'':
\begin{enumerate}
    \item Click the \textbf{``Inject Demo Threat''} button on the dashboard sidebar.
    \item Watch the threat appear in the Stream table within seconds.
    \item Observe the toast notification pop up.
    \item Show the new \textbf{CRITICAL} alert card with Gemini's AI analysis.
\end{enumerate}

\subsection{Step 4: Explain the AI Analysis (30 seconds)}
Zoom in on the \textbf{Gemini Insight} box:
\begin{itemize}
    \item Show how the AI explains the threat in plain English.
    \item Highlight the actionable recommendation (e.g., ``Update Nginx to v1.19+'').
\end{itemize}

\subsection{Step 5: Ask the Chatbot (Optional)}
In the sidebar, ask SENTINEL:
\begin{quote}
``What should I do about this nginx vulnerability?''
\end{quote}
Show the AI-powered response utilizing the knowledge base.

\section{Key Talking Points}

\begin{enumerate}
    \item \textbf{Speed}: ``From NIST publication to alert in under 5 seconds.''
    \item \textbf{Filtering}: ``Only alerts on YOUR inventory, not every CVE.''
    \item \textbf{AI Context}: ``Gemini explains WHY it matters, not just the score.''
    \item \textbf{Real-Time}: ``No refresh button needed - it's truly streaming.''
\end{enumerate}

\section{Fallback: Manual Injection}

If the dashboard button doesn't work, inject via terminal:
\begin{lstlisting}[language=bash]
echo "CRITICAL: Zero-day exploit in production server" > manual_input.txt
\end{lstlisting}

\end{document}
